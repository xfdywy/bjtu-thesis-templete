		
		\chapter{引言}
		
	 
		
		
		\section{研究背景}
		
		
 
		\begin{figure}[h]
			\centering 
			\includegraphics[width=3in,height=2in]{rlp}
	   % 	\caption{  PSI transformation for RNN}
			\bicaption{  序列决策问题示例}{Illustration for sequential decision making problem}
			\label{fig:chap1 rlp}
		\end{figure}

  
  
  
		
		
		\section{本文的结构安排}
		文本的结构安排如下:
 


	 \section{参考文献示例}
参考文献bib文件放在reference文件夹中,在主文件demo中引用了bib文件,在所有chap中都可以直接cite。比如\cite{silver2016mastering}

\section{算法框示例}

\begin{algorithm}[h]
	\caption{ 优化算法}
	\label{gtdmc:alg:minmax}			
	\KwIn{ 迭代次数$ T$,   待学习参数初始值$x_1$ 、 $y_1$ , 学习率 $\alpha$ }
	\For{$ t = 1, \dots, T $}{
		更新参数:  
		
		$ y_{t+1} = \mathcal{P}_{\mathcal{X}_y}\left(y_t + \alpha_t(\hat{b}_t - \hat{A}_t x_t -\hat{M}_ty_t)\right)  $
		
		$ x_{t+1} = \mathcal{P}_{\mathcal{X}_x}\left(x_t + \alpha_t\hat{A}_t^\top y_t\right) $
	}
	
	
	\KwOut{ 		$ \quad 	 \tilde{x}_T = \frac{\sum_{t=1}^{T}\alpha_t x_t}{\sum_{t=1}^{T}\alpha_t}  \qquad \tilde{y}_T = \frac{\sum_{t=1}^{T}\alpha_t y_t}{\sum_{t=1}^{T}\alpha_t} $	 }				
	
	
\end{algorithm}	 

\section{图片示例}



\begin{figure}[h]
	\centering
	\subfloat[a]{
		\label{figa}
		\includegraphics[width=1.8in,height=1.5in]{a.png}}
	\subfloat[b]{
		\label{figc}
		\includegraphics[width=1.8in,height=1.5in]{a.png}} 
	\subfloat[c ]{
		\label{fige}
		\includegraphics[width=1.8in,height=1.5in]{a.png}}	
	
	
	\subfloat[d ]{
		\label{figb}
		\includegraphics[width=1.8in,height=1.5in]{a.png}}
	\label{Fig1}	
	\subfloat[e ]{
		\label{figd}
		\includegraphics[width=1.8in,height=1.5in]{a.png}}	
	\subfloat[f ]{
		\label{figf}
		\includegraphics[width=1.8in,height=1.5in]{a.png}}		
	\bicaption{你好,世界}{Hello world } 
	\label{fig1}
\end{figure}	
