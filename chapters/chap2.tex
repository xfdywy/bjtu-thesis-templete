\chapter{第二章}
[鼠标左键单击选择该段落,输入替换之。内容为小四号宋体。] 学位论文为了需要反映出作者确已掌握了坚实的基础理论和系统的专门知识,具有开阔的科学视野,对研究方案作了充分论证,因此,有关历史回顾和前人工作的综合评述,以及理论分析等,可以单独成章,用足够的文字叙述。正文是学位论文的核心部分,占主要篇幅,可以包括:调查对象、实验和观测方法、仪器设备、材料原料、实验和观测结果、计算方法和编程原理、数据资料、经过加工整理的图表、形成的论点和导出的结论等。

\section{第一节}

由于研究工作涉及的学科、选题、研究方法、工作进程、结果表达方式等有很大的差异,对正文内容不能作统一的规定。但是,必须实事求是,客观真切,准确完备,合乎逻辑,层次分明,简练可读。

\subsection{第一小节}

图:见《北京交通大学学位论文撰写规范》3.10.4
图应有编号。图的编号由“图”和从“1”开始的阿拉伯数字组成,图较多时,可分章依序编号。
图宜有图题,图题即图的名称,置于图的编号之后。图的编号和图题应置于图下方。图题采用中英文对照,英文(Times New Roman)字体五号,中文宋体五号。居中书写,中文在上。
照片图要求主题和主要显示部分的轮廓鲜明,便于制版。如用放大缩小的复制品,必须清晰,反差适中。照片上应有表示目的物尺寸的标度。
表:见《北京交通大学学位论文撰写规范》3.10.5
表应有编号。表的编号由“表”和从“1”开始的阿拉伯数字组成,表较多时,可分章依序编号。
表宜有表题,表题即表的名称,置于表的编号之后。表的编号和表题应置于表上方。表题采用中英文对照,居中,中文在上。英文(Times New Roman)字体五号,中文宋体五号。
表的编排,一般是内容和测试项目由左至右横读,数据依序竖读。
表的编排建议采用国际通行的三线表。
如某个表需要转页接排,在随后的各页上应重复表的编号。编号后跟表题(可省略)和“(续)”,置于表上方。
续表均应重复表头。
公式:见《北京交通大学学位论文撰写规范》3.10.12
论文中的公式应另行起,并居中书写,与周围文字留足够的空间区分开。
如有两个以上的公式,应用从“1”开始的阿拉伯数字进行编号,并将编号置